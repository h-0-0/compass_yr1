% !TEX root = /home/jd18380/Documents/compass_yr1/Stat_methods/Portfolio/Section_2/main.tex
\documentclass[a4paper]{article}

%% Language and font encodings
\usepackage[english]{babel}
\usepackage[utf8x]{inputenc}

\usepackage{booktabs}
\usepackage{tabu}
\usepackage[T1]{fontenc}
\usepackage{subfig}

%% Sets page size and margins
\usepackage[a4paper,top=2cm,bottom=3cm,left=3cm,right=3cm,marginparwidth=2.75cm]{geometry}
%% Useful packages
\usepackage{amsmath}
\usepackage{amsfonts}
\usepackage{amssymb}
\usepackage{bm}
\usepackage{xcolor}
\DeclareMathOperator*{\argmax}{arg\!\max}
\DeclareMathOperator*{\argmin}{arg\!\min}
\usepackage[toc,page]{appendix}
\usepackage{graphicx}
%\usepackage{apacite}
\usepackage[colorinlistoftodos]{todonotes}
\usepackage[colorlinks=true, allcolors=blue]{hyperref}
\usepackage{cleveref}

\newtheorem{theorem}{Theorem}[section]
\newtheorem{corollary}{Corollary}[theorem]
\newtheorem{lemma}[theorem]{Lemma}
\newtheorem{definition}{Definition}[section]
\newtheorem{question}{Question}[subsection]

\newcommand{\yi}{y_{i}}
\newcommand{\gxi}{g_{\bm{x}_{i}}}
\newcommand{\gx}{g_{\bm{x}}}
\newcommand{\ei}{\epsilon_{i}}
\newcommand{\E}{\mathbb{E}}
\newcommand{\wls}{\bm{w}_{LS}}
\newcommand{\wlsr}{\bm{w}_{LS-R}}
\newcommand{\wmap}{\bm{w}_{MAP}}
\newcommand{\xui}{\bm{x}_{i}}
\newcommand{\prls}{f(\xui ; \wls)}
\newcommand{\prlsv}{f(\bm{x} ; \wls)}
\newcommand{\V}{\text{Var}}
\newcommand{\y}{\bm{y}}
\newcommand{\x}{\bm{x}}
\newcommand{\w}{\bm{w}}
\newcommand{\F}{\bm{\phi}}
\newcommand{\I}{\bm{I}}
\newcommand{\X}{\bm{X}}
\newcommand{\K}{\bm{K}}
\newcommand{\yh}{\hat{y}}
\newcommand{\dpost}{\Delta \textit{posterior}}
\newcommand{\dprior}{\Delta \textit{prior}}


\title{Statistical Methods: Portfolio 2}
\author{By Henry Bourne}
\date{}

\begin{document}
\maketitle


% \begin{abstract}
%     In this document we will summarize content from the first 4 lectures from the Statistical methods course (at Compass, University of Bristol). These lectures cover content on using statistical methods for decision-making.
% \end{abstract}

\section{Linear Classifiers}
In the portfolio 1 we saw how to conduct binary classification, now we will look at how to conduct \textbf{multi-class classification}. This is where we have an input $\x \in \mathbb{R}^{d}$ and an output $y \in \{1,...,K\}$.

The geometry of the problem in this case is more complicated than we had with binary classification, we can no longer simply check the sign of a single $f(\x)$ to classify. We could try introducing multiple functions. Let's say we have 3 classes, we could try introducing another function so now we have $f_{1}$ and $f_{2}$, we could then perform classification by checking the signs of both $f_{1}$ and $f_{2}$ and have this dictate the class. However, this can get confusing as for example in this case we have 4 possible outcomes, $\{+,-\}^{2}$, for the signs of our f's, however, we only have 3 classes. What about if we introduce pairwise binary classifiers, ie. we have $f_{i,j}$ classifies a point as either i or j and we have such a function for any pair of classes, then we classify a point as the majority vote given by the binary classifiers. The problem here is that all the classifiers may disagree in which case there is no majority vote.

Rather than relying on the sign of a function f to make predictions lets instead fit a vector valued function $\bm{f}: \mathbb{R}^{d} \rightarrow \mathbb{R}^{K}$, where K is the number of classes. Given an input $\x$ our prediction is $\hat{k} = \argmax_{k} f^{(k)} (\x)$. Note that this no longer has a simple geometric interpretation anymore. 

\subsection{Least Squares Classifier}
We will first define the least squares classifier in the binary classification case and after extend it to the multi-class case:
\begin{definition}
    \textbf{Least Squares Binary Classifier:} \\
    We first perform LS on the data, ie. find:
    \begin{equation}
        \wls := \argmin_{\w} \sum_{i \in D} [y_{i} - f(\xui ; \w)]^{2}
    \end{equation}
    Then we can find the predicted label $\hat{y}:= \text{sign}(f(\xui;\wls))$
\end{definition}
Note that we can also use a feature transform for f as well, which would allow us to fit more complex classifiers, as data not separable in the original space may be separable in the feature space. In the multi-class case:
\begin{definition} \label{def:multi-class-LS-class}
    \textbf{Multi-class LS classification:} \\
    We use a \textbf{one-hot encoding} which is where we replace $y_{i}=k$ in our data with $\t_{i} \in \{0,1\}^{K}$ where all entries are 0 bar $t_{i}^{(k)}=1$. Then we have:
    \begin{equation}
        \Wls := \argmin_{\W} \sum_{i \in D} ||\t_{i} - f(\xui ; \W)||^{2}
    \end{equation}
    where $\W \in \mathbb{R}^{(d+1) \times K}$, $f(\x;\W)=\W^{T} \tilde{\x}$ and $\tilde{\x} := [\x^{T}, 1]^{T}$.\\
    Then our prediction $\hat{y} = \argmax_{k} f(\x;\W)^{(k)} = \argmax_{k}(\wls^{(k)})^{T} \tilde{\x}$, where $\w^{(k)}$ is the k-th column of $\W$.
\end{definition}
Although this method can work the square loss tends not to make sense in a classification task, as a point far away from the boundary (fit without that point) can dramatically affect our decision boundary, even if it is correctly classified by the decision boundary (fit without that point). Also, unlike with LS regression, LS classification lacks a probabilistic interpretation.

\subsection{Fisher Discriminant Analysis (FDA)}
Note that taking the inner product $\langle \w, \x \rangle$ embeds $\x$ onto a one-dimensional line along the $\w$ direction. We can say $\w$ gives a good embedding if the $\x$ it embeds are close together in the embedding if they are from the same class but far apart if they are from different classes. We can define these two properties we want from our embedding as:
\begin{definition}
    \textbf{Within-class Scatterness:} \\
    For embedding $\w^{T} \x$, the \textbf{embedding centre} for class k is:
    \begin{equation}
        \muhatk = \frac{1}{n_{k}} \sum_{i, y_{i}=k} \w^{T} \xui
    \end{equation}
    then the within-class scatterness of class k is:
    \begin{equation}
        s_{\w,k} := \sum_{i,y_{i}=k} (\w^{T} \xui - \muhatk)^{2}
    \end{equation}
\end{definition}
\begin{definition}
    \textbf{Between-class Scatterness:} \\
    The \textbf{embedded dataset centre} is:
    \begin{equation}
        \hat{\mu} = \frac{1}{n} \sum_{i=1}^{n} \w^{T} \xui
    \end{equation}
    then teh between-class scatterness is:
    \begin{equation}
        s_{b,k} = n_{k} (\muhatk - \hat{\mu})^{2}
    \end{equation}
    note the $n_{k}$ is needed to make $s_{b,k}$ the same scale as $s_{w,k}$.
\end{definition}
Then ideally we would like to maximize the between-class scatterness and minimize the within-class scatterness for all the classes, which is what the following does:
\begin{definition}
    \textbf{Fisher Discriminant Analysis (FDA):} \\
    \begin{equation}
        \max_{\w} [\sum_{k} s_{b,k} / \sum_{k} s_{\w,k} ]
    \end{equation}
\end{definition}
if $K=2$ then this has a simple solution: $\w := \bm{S}_{\w}^{-1} (\bm{\mu}_{+} - \bm{\mu}_{-})$, where $\bm{S}_{\w} := \sum_{k=1}^{K} \bm{S}_{k}$ and $\bm{S}_{k}$ is the sample covariance matrix of class k times $n_{k}$. However, note that the FDA does not learn a decision function f, the $\w_{FDA}$ obtained cannot be used directly by the prediction function to make a prediction. This is because (eg. in the binary case) $f(\x;\w_{FDA})>0$ does not mean that $\x$ is predicted as the positive class. The FDA also doesn't care about classification accuracy, ie. minimizing the FP or FN rate.

\subsection{Probabilistic (Generative) Classifiers}
How do we put a classification problem under a probabilistic framework? Let's say we would like to \textbf{minimize the expected loss:} $\hat{y} := \argmin_{y_{0}} \E_{p(y|\x)} [L(y,y_{0})|\x]$. To do this we need $p(y|\x)$. The \textbf{discriminative approach} would be to infer $p(y|\x)$ directly. Here, we will look at the \textbf{generative approach} which is where we note that $p(y|\x) \propto p(\x|y) p(y)$ and we infer $p(\x|y)$ \footnote{$p(y)$ is just the probability of the class being class y}.

\subsubsection{Continuous input} \label{subsubsection:Continuous input}
If $\x$ is a continuous variable then the Multi-Variate Normal (MVN) is a natural choice for the model of $p(\x|y)$, ie. we have $p(\x | y=k;\w) := N_{\bm{x}}(\bm{\mu_{k}}, \bm{\Sigma_{k}})$ (assuming iid data and that all classes have same covariance matrix). We can write the likelihood of the parameters over the data, D, as: $p(D|\w) = \prod_{i \in D}p(\xui , y_{i}|\w) = \prod_{i \in D} p(\xui | y_{i} ;\w)p(y_{i}) = \prod_{i \in D} N_{\xui} (\bm{\mu}_{y_{i}};\bm{\Sigma}) p(y_{i})$. Therefore our MLE's are: $\hat{\bm{\mu}_{1}, ..., \bm{\mu}_{K},\hat{\bm{\Sigma}}} := \argmax_{\bm{\mu}_{1}, ..., \bm{\mu}_{K},\hat{\bm{\Sigma}}} \sum_{i \in D} \log [ N_{\xui} (\bm{\mu}_{y_{i}} ; \bm{\Sigma}) p(y_{i})]$. If we plug in our estimates for $p(y_{i}=k)$ which are $\frac{n_{k}}{n}$, then we can find the MLE for $\hat{\bm{\mu}}_{k} := \frac{1}{n_{k}} \sum_{i \in D, y_{i}=k} \xui$. We can then plug in our $\hat{\bm{\mu}}_{k}$'s to work out: $\hat{\bm{\Sigma}} := \sum_{k=1,...,K} \frac{n_{k}}{n} \color{red} \frac{1}{n_{k}} \sum_{i \in D, y_{i}=k} (\xui - \bmuhatk)(\xui - \bmuhatk)^{T} \color{black}$, where we notice that in red is the MLE of the covariance of individual classes. Our prediction is $\hat{y} := \argmax_{y} p(y| \x ; \hat{\w}) \propto p(\x | y ; \hat{\w}) p(y)$. We can also prove that when using the shared covariance matrix MVN model, the decision boundary is piecewise-linear (\Cref{Proofs:linear-decision-boundary}).

What if we assume that for each class k there are different covariance matrices? then the MLE reduces to estimating individual $\bm{\mu}_{k}$ and $\bm{\Sigma}_{k}$. Note that in this case the decision boundary is no longer linear.

\subsubsection{Discrete input}
Here we will look at the case where $\x$ is discrete. Let $\x := [x^{(1)},...,x^{(d)}]^{T}$ and assume $x^{(1)},...,x^{(d)}$ follow a multinomial distribution, where each $x^{(i)}$ is a quantity for a feature. Using Bayes rule we have that $p(y=k| \x) = (p(\x|y=k) \cdot p(y=k)) / p(\x)$. We now introduce our "naive assumptions" which are that all the features in $\x$ are mutually independent, conditional on $y=k$. Using these assumptions we can write $p(y=k | \x) \propto p(y=k) \prod_{i=1}^{n} p(x_{i} |y=k)$ \footnote{We get this by discarding the $p(\x)$, which makes it proportional and then using our assumption on $p(\x|y=k)$}. Therefore $p(\x = \x_{0} |y=k) \propto \prod_{i=1,...,d} \beta(i|y=k)^{x_{0}^{(i)}}$, for some $\x_{0}$ where $\beta(i|y=k)$ is the probability feature i occurs in class k. It is easy to estimate this quantity:
\begin{equation}
    \beta(i|y=k) \approx \frac{\sum_{j \in D, y_{j}=k} x_{j}^{(i)}}{\sum_{j \in D, y_{j}=k} \sum_{l=1}^{d} x_{j}^{(l)}}
\end{equation}
Then our prediction is $\hat{y} := \argmax_{y} p(\x=\x_{0} |y) \cdot p(y)$ where as before $p(y=k)$ can be estimated as $\frac{n_{k}}{n}$ and note that $\beta(i|y=k)$ can be estimated by counting. This is called \textbf{Naive Bayes classification"}.
\section{Feature Transforms and Kernel Methods}
We will start by discussing different types of feature transforms before diving into kernel methods.

\subsection{Feature transforms}
We've already seen polynomial feature transforms, but why do they work? Let's get some intuition from the the 1-dimensional case. The taylor series of the data generating function, g, at zero is: $g(x) = g(0) (x-0)^{0} +g'(0)(x-0)^{1} + \frac{g''(0)}{2!}(x-0)^{2} + ...$. The taylor series tells us that we can approximate a smooth function using polynomial terms (at some cost), this gives some intuition for how a polynomial transform could allow us to create a good model of the data generating function. Another way we can decompose g is using the \textbf{Fourier series}: $g(x) = a_{0} + \sum_{i=1}^{\infty} [a_{i} \sin(ix) + b_{i} \cos(ix)]$. We can use the fourier series as intuition for another type of transform:
\begin{definition}
    \textbf{Trigonometric transform:} \\
    Used to approximate g(x) over the time domain, defined as follows,
    \begin{equation}
        \phi(\x) := [\sin(\x), \cos(\x), sin(2\x), cos(2\x), ..., \sin(b\x), \cos(b\x)]
    \end{equation}
\end{definition}
polynomial and trigonometric transforms are based on the idea a function can be approximated by a:
\begin{definition}
    \textbf{Linear basis expansion:} \\
    A linear basis expansion of $g(\x)$ is,
    \begin{equation}
        g(\x) \approx f(\x; \bm{w}) = \langle \bm{w} , \phi(\x) \rangle = \sum_{i=1}^{b} w^{(i)} \phi^{(i)} (\x)
    \end{equation}
    where $\phi^{(i)}$ are called \textbf{basis functions}
\end{definition}
A widely used basis function for regression tasks is the,
\begin{definition}
    \textbf{Radial Basis Function (RBF):}
    \begin{equation}
        \phi^{(i)}(\x) := \exp(- \frac{|| \x - \xui ||^{2}}{2 \sigma^{2}})
    \end{equation}
    where $\sigma >0$ is called the \textbf{bandwidth}. Note that $\sigma$ is determined before fitting, a common practice is setting it equal to the median of all pairwise distances of $\x$ in your dataset. \\
    $\xui$ are called \textbf{RBF centroids} (often chosen randomly from dataset). We will use $\bm{\phi}(\x)$ to denote $[\phi^{(1)}(\x),...,\phi^{(b)}(\x)]$
\end{definition}
If we imagine the prediction function starting as a flat line, if $w^{(i)}>0$ then the RBF at at $\xui$ lifts up the value of the prediction function around $\xui$, we have the converse is true if $w^{(i)}<0$. Each RBF defines a ball on which the prediction function is supported (with radius $\sigma$), if $g(\x)$ has a wide support then the prediction function must be supported almost everywhere and need many centroids. The packing number (of the number of balls) $b=O(c^{d})$, where $d$ is dimension of $\x$. 

\subsection{Kernel methods}
The larger the value of b, the higher the dimensionality of the feature space and the more flexible our prediction function is. Can we have an $\infty$-dimensional feature space? 

Suppose $\phi(\x)$ maps $\x$ to an $\infty$-dim. feature space, we will have that $\w$ will be $\infty$ly long. However, we can show that $\wlsr := (\F \F^{T} + \lambda \I)^{-1} \F \y^{T} = \F ( \F^{T} \F + \lambda \I)^{-1} \y^{T}$, where $\F$ is short for $\phi(\X)$ (proof in \cref{proof:wlsr-woodbury}). We now no longer need to compute $\F \F^{T}$ (intractable), we instead can compute $\F^{T} \F \in \mathbb{R}^{n*n}$.

Is there a way we can compute the inner-product without computing $\phi(\cdot)$? Define $k(\bm{a},\bm{b}) := \langle \phi(\bm{a}), \phi(\bm{b}) \rangle$, denote $\K := \F^{T} \F $. We have that $K^{(i,j)} = \langle \phi(\xui), \phi(\bm{x}_{j}) \rangle = k(\xui, \bm{x}_{j})$. Going back to our prediction function we have that, $f(\x ; \wlsr) = \langle \wlsr , \phi(\x) \rangle = \langle \F(\K + \lambda I)^{-1} \y^{T} , \phi(\x) \rangle = \langle (\K+\lambda \I)^{-1} \y^{T}, \phi(\x)^{T} \F \rangle $. Let's denote $\bm{k} := \phi(\x)^{T} \F$, where $k^{(i)} = \langle \phi(\x), \phi(\xui) \rangle = k(\x, \xui)$. We can then rewrite our prediction functions as $f(\x ; \wlsr) := \bm{k} (\K + \lambda \I)^{-1} \y^{T}$, note that $\phi(\x)$ only appears in inner products! \footnote{$\K=O(n^{2})$ and $(\K + \lambda \I)^{-1}$ is usually $O(n^{3})$, so computational cost gets very large for a large n}

All we need now is a function, k, that mimics the behavior of the inner product (without actually having to compute the inner product) and if our $k(\bm{a}, \bm{b})$ is positive definite $\exists \phi$ such that $k(\bm{a}, \bm{b}) = \langle \phi(\bm{a}), \phi(\bm{b}) \rangle$. Such a function is called a:
\begin{definition}
    \textbf{Kernel function:} \\
    Function $k(\bm{a}, \bm{b})$ that mimics the inner product, if explicit $\phi(\x)$ can be derived from k we say k induces feature transform $\phi(\x)$. Many known choices of k exist.
\end{definition}

\subsection{Choosing a kernel function, k}
First we give some examples of kernel functions,
\begin{definition}
    \textbf{Linear Kernel Function:} \\
    $k(\bm{a}, \bm{b}) := \langle \bm{a}, \bm{b} \rangle$, induces feature transform $\phi(\x) = \x$
\end{definition}
\begin{definition}
    \textbf{Polynomial Kernel Function with Degree b:} \\
    $k(\bm{a}, \bm{b}) := (\langle \bm{a} , \bm{b} \rangle +1)^{b}$, this induces feature transform $\phi(\x) = [\bm{x}, ..., \bm{x}^{b}]$.
\end{definition}
\begin{definition}
    \textbf{RBF (or Gaussian) Kernel} \footnote{Note: RBF basis function and the RBF kernel are \textbf{not} the same thing.} \textbf{:} \\
    $k(\bm{a}, \bm{b}) := \exp(- \frac{|| \bm{a} - \bm{b} ||^{2}}{2 \sigma^{2}})$, induces feature transform $\phi(\x)$ that is infinite-dimensional, $\sigma$ is chosen before fitting (often as the median of the pairwise distances of all inputs).
\end{definition}
How do we choose an appropriate kernel function? it depends on the learning task and dataset. The RBF kernel is a good all-round choice for $\x \in \mathbb{R}^{d}$.
\section{Support Vector Machines}\label{section:Support-Vector-Machines}
In binary classification it might be that there are many decision boundaries we could draw that would correctly classify all the training data, this leaves us asking: what is the "optimal" decision boundary in binary classification? For our decision boundary to be generalizable we want to minimize the error on unseen datasets rather than the training set. We would like a decision boundary that will correctly classify a small perturbation of one of our training data points whilst still correctly classifying all of our training data, ie. we want to maximize the gap (the \textbf{margin}) between the decision boundary and the lines given by $f(\x;\w) = 1$ and $f(\x;\w) = -1$, whilst keeping data points on the correct side of the margin. 

If we are using decision boundary $f(\x;\w) := \langle\w_{1},\x\rangle + w_{0}$ then we can calculate the thickness of our margin as $\frac{1}{||\w_{1}||}$. Then we can formulate the following constrained minimization problem: minimize $||\w_{1}||^{2}$ subject to $\forall i, y_{i} \cdot f(\xui;\w) \geq 1$.

However, in many cases the dataset is not separable and so it is not possible to satisfy our constraint. We can relax these constraints which is what we do in the \textbf{soft-margin classifier}: $\min_{\w, \bm{\epsilon}} ||\w_{1}||^{2} + \sum_{i} \epsilon^{(i)}$ subject to $\forall i , \: y_{i}(\langle\w_{1},\xui \rangle + w_{0}) + \epsilon^{(i)} \geq 1, \: \epsilon^{(i)} \geq 0$. It turns out that this classifier is actually a convex minimization problem and so every local minimum is a global minimum.

We can use the lagrangian dual (discussed in \cref{Proofs:lagrangian-dual}) to turn the minimization problem of the soft-margin classifier into: $\max_{\bm{\lambda}} - \frac{\tilde{\bm{\lambda}}^{T} \X^{T} \X \tilde{\bm{\lambda}}}{4} + \langle\bm{\lambda},\bm{1} \rangle$ subject to $0 \leq \lambda_{i} \leq 1, \: \sum_{i} \lambda_{i} y_{i} =0$. Note that the constraints here are simpler and that its quadratic wrt. $\bm{\lambda} \in \mathbb{R}^{n}$ as opposed to quadratic wrt. $\w \in \mathbb{R}^{d+1}$ in our previous optimization problem. Hence, the lagrangian dual is slow when n is large and our previous problem is slow when d is large, also note that it is possible to use kernel methods in solving the lagrangian!

Notice that the border of the (optimal) margins will always pass through some data points, these points can be thought of as resisting the expansion of the margin, hence are called \textbf{support vectors}. Hence the name of this method: "Support Vector Machines" (SVM). However, the SVM has some limitations, for example it is important to note the SVM is not a probabilistic classifier as it lacks a probabilistic interpretation. Also note that the computational cost of SVM is high as it involves solving a constrained optimization problem (as opposed to unconstrained). Finally multi-class SVM is non-trivial as SVM is motivated by the geometry of binary classification. 
\newpage
\begin{appendices}


\section{Proofs}

\subsection{} \label{}



\newpage
\section{Homeworks}
\subsection{For Section 1}
\begin{question}

\end{question}


\end{appendices}

% \small
% \bibliographystyle{plain}
% \bibliography{refs}

\end{document}